% Options for packages loaded elsewhere
\PassOptionsToPackage{unicode}{hyperref}
\PassOptionsToPackage{hyphens}{url}
%
\documentclass[
]{article}
\usepackage{lmodern}
\usepackage{amssymb,amsmath}
\usepackage{ifxetex,ifluatex}
\ifnum 0\ifxetex 1\fi\ifluatex 1\fi=0 % if pdftex
  \usepackage[T1]{fontenc}
  \usepackage[utf8]{inputenc}
  \usepackage{textcomp} % provide euro and other symbols
\else % if luatex or xetex
  \usepackage{unicode-math}
  \defaultfontfeatures{Scale=MatchLowercase}
  \defaultfontfeatures[\rmfamily]{Ligatures=TeX,Scale=1}
\fi
% Use upquote if available, for straight quotes in verbatim environments
\IfFileExists{upquote.sty}{\usepackage{upquote}}{}
\IfFileExists{microtype.sty}{% use microtype if available
  \usepackage[]{microtype}
  \UseMicrotypeSet[protrusion]{basicmath} % disable protrusion for tt fonts
}{}
\makeatletter
\@ifundefined{KOMAClassName}{% if non-KOMA class
  \IfFileExists{parskip.sty}{%
    \usepackage{parskip}
  }{% else
    \setlength{\parindent}{0pt}
    \setlength{\parskip}{6pt plus 2pt minus 1pt}}
}{% if KOMA class
  \KOMAoptions{parskip=half}}
\makeatother
\usepackage{xcolor}
\IfFileExists{xurl.sty}{\usepackage{xurl}}{} % add URL line breaks if available
\IfFileExists{bookmark.sty}{\usepackage{bookmark}}{\usepackage{hyperref}}
\hypersetup{
  pdftitle={Overview of IHDS 2012 Education Data},
  hidelinks,
  pdfcreator={LaTeX via pandoc}}
\urlstyle{same} % disable monospaced font for URLs
\usepackage[margin=1in]{geometry}
\usepackage{color}
\usepackage{fancyvrb}
\newcommand{\VerbBar}{|}
\newcommand{\VERB}{\Verb[commandchars=\\\{\}]}
\DefineVerbatimEnvironment{Highlighting}{Verbatim}{commandchars=\\\{\}}
% Add ',fontsize=\small' for more characters per line
\usepackage{framed}
\definecolor{shadecolor}{RGB}{248,248,248}
\newenvironment{Shaded}{\begin{snugshade}}{\end{snugshade}}
\newcommand{\AlertTok}[1]{\textcolor[rgb]{0.94,0.16,0.16}{#1}}
\newcommand{\AnnotationTok}[1]{\textcolor[rgb]{0.56,0.35,0.01}{\textbf{\textit{#1}}}}
\newcommand{\AttributeTok}[1]{\textcolor[rgb]{0.77,0.63,0.00}{#1}}
\newcommand{\BaseNTok}[1]{\textcolor[rgb]{0.00,0.00,0.81}{#1}}
\newcommand{\BuiltInTok}[1]{#1}
\newcommand{\CharTok}[1]{\textcolor[rgb]{0.31,0.60,0.02}{#1}}
\newcommand{\CommentTok}[1]{\textcolor[rgb]{0.56,0.35,0.01}{\textit{#1}}}
\newcommand{\CommentVarTok}[1]{\textcolor[rgb]{0.56,0.35,0.01}{\textbf{\textit{#1}}}}
\newcommand{\ConstantTok}[1]{\textcolor[rgb]{0.00,0.00,0.00}{#1}}
\newcommand{\ControlFlowTok}[1]{\textcolor[rgb]{0.13,0.29,0.53}{\textbf{#1}}}
\newcommand{\DataTypeTok}[1]{\textcolor[rgb]{0.13,0.29,0.53}{#1}}
\newcommand{\DecValTok}[1]{\textcolor[rgb]{0.00,0.00,0.81}{#1}}
\newcommand{\DocumentationTok}[1]{\textcolor[rgb]{0.56,0.35,0.01}{\textbf{\textit{#1}}}}
\newcommand{\ErrorTok}[1]{\textcolor[rgb]{0.64,0.00,0.00}{\textbf{#1}}}
\newcommand{\ExtensionTok}[1]{#1}
\newcommand{\FloatTok}[1]{\textcolor[rgb]{0.00,0.00,0.81}{#1}}
\newcommand{\FunctionTok}[1]{\textcolor[rgb]{0.00,0.00,0.00}{#1}}
\newcommand{\ImportTok}[1]{#1}
\newcommand{\InformationTok}[1]{\textcolor[rgb]{0.56,0.35,0.01}{\textbf{\textit{#1}}}}
\newcommand{\KeywordTok}[1]{\textcolor[rgb]{0.13,0.29,0.53}{\textbf{#1}}}
\newcommand{\NormalTok}[1]{#1}
\newcommand{\OperatorTok}[1]{\textcolor[rgb]{0.81,0.36,0.00}{\textbf{#1}}}
\newcommand{\OtherTok}[1]{\textcolor[rgb]{0.56,0.35,0.01}{#1}}
\newcommand{\PreprocessorTok}[1]{\textcolor[rgb]{0.56,0.35,0.01}{\textit{#1}}}
\newcommand{\RegionMarkerTok}[1]{#1}
\newcommand{\SpecialCharTok}[1]{\textcolor[rgb]{0.00,0.00,0.00}{#1}}
\newcommand{\SpecialStringTok}[1]{\textcolor[rgb]{0.31,0.60,0.02}{#1}}
\newcommand{\StringTok}[1]{\textcolor[rgb]{0.31,0.60,0.02}{#1}}
\newcommand{\VariableTok}[1]{\textcolor[rgb]{0.00,0.00,0.00}{#1}}
\newcommand{\VerbatimStringTok}[1]{\textcolor[rgb]{0.31,0.60,0.02}{#1}}
\newcommand{\WarningTok}[1]{\textcolor[rgb]{0.56,0.35,0.01}{\textbf{\textit{#1}}}}
\usepackage{graphicx,grffile}
\makeatletter
\def\maxwidth{\ifdim\Gin@nat@width>\linewidth\linewidth\else\Gin@nat@width\fi}
\def\maxheight{\ifdim\Gin@nat@height>\textheight\textheight\else\Gin@nat@height\fi}
\makeatother
% Scale images if necessary, so that they will not overflow the page
% margins by default, and it is still possible to overwrite the defaults
% using explicit options in \includegraphics[width, height, ...]{}
\setkeys{Gin}{width=\maxwidth,height=\maxheight,keepaspectratio}
% Set default figure placement to htbp
\makeatletter
\def\fps@figure{htbp}
\makeatother
\setlength{\emergencystretch}{3em} % prevent overfull lines
\providecommand{\tightlist}{%
  \setlength{\itemsep}{0pt}\setlength{\parskip}{0pt}}
\setcounter{secnumdepth}{-\maxdimen} % remove section numbering

\title{Overview of IHDS 2012 Education Data}
\author{}
\date{\vspace{-2.5em}}

\begin{document}
\maketitle

\hypertarget{overview-of-ihds-data}{%
\section{Overview of IHDS data}\label{overview-of-ihds-data}}

IHDS is a panel survey led by Sonalde Desai and others at NCAER/UofMd.
The first round of the panel was conducted in 2004-05 and the second
wave was conducted in 2011-12. In both rounds of the survey they
administered a test similar to the ASER test to all children aged 8-11
and age 15-18. In addition, the survey asked basic questions about
enrollment and education achievement for all household members.

\hypertarget{datasets-and-documentation}{%
\section{Datasets and documentation}\label{datasets-and-documentation}}

When you download the round 2 data, you get 14 datasets. These are split
into folders ``DS00X'' where X corrresponds to the following:

\begin{enumerate}
\def\labelenumi{\arabic{enumi}.}
\tightlist
\item
  Individual
\item
  Household
\item
  Eligible Women
\item
  Birth History 5.Medical Staff
\item
  Medical Facilities
\item
  Non Resident
\item
  School Staff
\item
  School Facilities
\item
  Wage and Salary
\item
  Tracking
\item
  Village
\item
  Village Panchayat
\item
  Village Respondent
\end{enumerate}

Note that, unlike the panel dataset, there is only one version of the
files. (For the panel dataset, there are different versions of the files
corresponding to different ways of merging/combining the two rounds of
data.) More information about each of the datasets, including which
variables to use for weighting observations when using each dataset and
how to merge the datasets, can be found
\href{https://www.icpsr.umich.edu/icpsrweb/content/DSDR/idhs-II-data-guide.html}{here}.
According to documentation, merge on stateid, distid, psuid, hh,
hhsplitid

Documentation for the 2012 dataset is saved in the ``IHDS/IHDS 2012''
folder in my data folder. The user guide contains basic information
about the sampling strategy and various constructed variables (e.g.~the
consumption variables). The data guide seems to just duplicate this
information. To figure out which variables to use, go to the codebooks
and questionnaires located in each of the

\hypertarget{finding-variables-in-the-datasets}{%
\subsubsection{Finding variables in the
datasets}\label{finding-variables-in-the-datasets}}

The fastest way to find relevant variables in the dataset is to first
search for keywords in the questionnaire to find the relevant question
number and then search for the question number in the codebook. This can
be a bit confusing since the variables in the codebook are organized by
subject rather than ordered by question number. (For example, some of
the education questions from the household roster are coded as ``CS''
questions and come later in the roster.)

Note that codebook section \textbf{``ED5: HQ19 11.5 Education: Enrolled
now''} refers to variable \textbf{ED5} in the dataset which comes from
\textbf{question 11.5} which is one \textbf{page 19} of the
\textbf{household questionnaire}.

\hypertarget{key-education-variables-in-the-individual-dataset-ds0001}{%
\section{Key education variables in the individual dataset
(DS0001)}\label{key-education-variables-in-the-individual-dataset-ds0001}}

I have included a list of some of the key education-related variables
(and other basic variables) below. Note that this list might not be
comprehensive (i.e.~there may be some ed related vars which I have
missed).

\hypertarget{identifiers-and-other-vars}{%
\subsubsection{Identifiers and other
vars}\label{identifiers-and-other-vars}}

\begin{itemize}
\tightlist
\item
  STATEID
\item
  PSUID
\item
  HHID
\item
  HHSPLITID
\item
  PERSONID
\item
  IDPSU - this one might be unique, not sure
\item
  WT - weights
\item
  RO3 - sex
\item
  RO7 - primary activity status
\item
  RO5 - Age
\end{itemize}

\hypertarget{testing-vars-for-children-8-11-page-38-of-questionnaire}{%
\subsubsection{Testing vars for children 8-11 (page 38 of
questionnaire)}\label{testing-vars-for-children-8-11-page-38-of-questionnaire}}

\begin{itemize}
\tightlist
\item
  TA8A, TA8B, TA9A, TA9B, TA10A, TA10B -- Reading, math, and writing
  test results (and the language they were administered in)
\item
  TA* - other vars related to ASER test
\end{itemize}

\hypertarget{education-vars-for-all-members-page-23-of-questionnaire}{%
\subsubsection{Education vars for all members (page 23 of
questionnaire)}\label{education-vars-for-all-members-page-23-of-questionnaire}}

\begin{itemize}
\tightlist
\item
  ED2-ED12 -- ed-related info such as highest grade, current enrollment,
  etc for all household members.
\end{itemize}

\hypertarget{education-vars-all-hh-members-who-were-enrolled-in-school-in-previous-12-months-page-44-of-questionnaire}{%
\subsubsection{Education vars all hh members who were enrolled in school
in previous 12 months (page 44 of
questionnaire)}\label{education-vars-all-hh-members-who-were-enrolled-in-school-in-previous-12-months-page-44-of-questionnaire}}

\begin{itemize}
\tightlist
\item
  CS3 - in school
\item
  CS4 - type of school
\item
  CS5 - distance to school
\item
  CS6 - standard years
\item
  CS8 - medium of instruction
\item
  CS9 - year English taught
\item
  CS10 - school hours per week
\item
  CS11 - homework hours per week
\item
  CS12 - private tuition hours per week
\item
  CS13 - days / months absent
\item
  (and many more)
\end{itemize}

\hypertarget{education-vars-for-children-8-11-page-46-of-questionnaire}{%
\subsubsection{Education vars for children 8-11 (page 46 of
questionnaire)}\label{education-vars-for-children-8-11-page-46-of-questionnaire}}

\begin{itemize}
\tightlist
\item
  CH*
\end{itemize}

\hypertarget{sample-code-to-read-in-key-variables}{%
\section{Sample code to read in key
variables}\label{sample-code-to-read-in-key-variables}}

\begin{Shaded}
\begin{Highlighting}[]
\CommentTok{# open the individual dataset (36151-0001) and select key education variables}
\KeywordTok{library}\NormalTok{(tidyverse)}
\end{Highlighting}
\end{Shaded}

\begin{verbatim}
## -- Attaching packages --------------------------------------------------- tidyverse 1.3.0 --
\end{verbatim}

\begin{verbatim}
## v ggplot2 3.2.1     v purrr   0.3.3
## v tibble  2.1.3     v dplyr   0.8.3
## v tidyr   1.0.0     v stringr 1.4.0
## v readr   1.3.1     v forcats 0.4.0
\end{verbatim}

\begin{verbatim}
## -- Conflicts ------------------------------------------------------ tidyverse_conflicts() --
## x dplyr::filter() masks stats::filter()
## x dplyr::lag()    masks stats::lag()
\end{verbatim}

\begin{Shaded}
\begin{Highlighting}[]
\KeywordTok{library}\NormalTok{(haven)}
\NormalTok{ihds_ind_dir <-}\StringTok{ "C:/Users/dougj/Documents/Data/IHDS/IHDS 2012/DS0001"}
\NormalTok{ind_file <-}\StringTok{ }\KeywordTok{file.path}\NormalTok{(ihds_ind_dir, }\StringTok{"36151-0001-Data.dta"}\NormalTok{)}
\CommentTok{# read in just those variables that i need}
\CommentTok{# this is much faster than reading in everything and then selecting}
\NormalTok{df <-}\StringTok{ }\KeywordTok{read_dta}\NormalTok{(ind_file, }\DataTypeTok{col_select =} \KeywordTok{c}\NormalTok{(STATEID, PSUID, HHID, HHSPLITID, PERSONID, IDPSU, WT, RO3, RO7, RO5, }\KeywordTok{starts_with}\NormalTok{(}\StringTok{"CS"}\NormalTok{), }\KeywordTok{starts_with}\NormalTok{(}\StringTok{"TA"}\NormalTok{), }\KeywordTok{starts_with}\NormalTok{(}\StringTok{"ED"}\NormalTok{)) )}
\end{Highlighting}
\end{Shaded}

\end{document}
